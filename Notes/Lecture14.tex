%%%%%%%%%%%%%%%%%%%%%%%%%%%%%%%%%%%%%%%%%%%%%%%%%%%%%%%%%%%%%%%%%%%%%%%%%%%%%%%%%%%%%%%%%%%%%%%%%%%%%%%%%%%%%%%%%%%%%%%%%%%%%%%%%%%%%%%%%%%%%%%%%%%%%%%%%%%%%%%%%%%
% Written By Michael Brodskiy
% Class: Fundamentals of Electronics
% Professor: M. Onabajo
%%%%%%%%%%%%%%%%%%%%%%%%%%%%%%%%%%%%%%%%%%%%%%%%%%%%%%%%%%%%%%%%%%%%%%%%%%%%%%%%%%%%%%%%%%%%%%%%%%%%%%%%%%%%%%%%%%%%%%%%%%%%%%%%%%%%%%%%%%%%%%%%%%%%%%%%%%%%%%%%%%%

\documentclass[12pt]{article} 
\usepackage{alphalph}
\usepackage[utf8]{inputenc}
\usepackage[russian,english]{babel}
\usepackage{titling}
\usepackage{amsmath}
\usepackage{graphicx}
\usepackage{enumitem}
\usepackage{amssymb}
\usepackage[super]{nth}
\usepackage{everysel}
\usepackage{ragged2e}
\usepackage{geometry}
\usepackage{multicol}
\usepackage{fancyhdr}
\usepackage{cancel}
\usepackage{siunitx}
\usepackage{physics}
\usepackage{tikz}
\usepackage{mathdots}
\usepackage{yhmath}
\usepackage{cancel}
\usepackage{color}
\usepackage{array}
\usepackage{multirow}
\usepackage{gensymb}
\usepackage{tabularx}
\usepackage{extarrows}
\usepackage{booktabs}
\usepackage{lastpage}
\usetikzlibrary{fadings}
\usetikzlibrary{patterns}
\usetikzlibrary{shadows.blur}
\usetikzlibrary{shapes}

\geometry{top=1.0in,bottom=1.0in,left=1.0in,right=1.0in}
\newcommand{\subtitle}[1]{%
  \posttitle{%
    \par\end{center}
    \begin{center}\large#1\end{center}
    \vskip0.5em}%

}
\usepackage{hyperref}
\hypersetup{
colorlinks=true,
linkcolor=blue,
filecolor=magenta,      
urlcolor=blue,
citecolor=blue,
}


\title{Lecture 14}
\date{\today}
\author{Michael Brodskiy\\ \small Professor: M. Onabajo}

\begin{document}

\maketitle

\begin{itemize}

  \item Metal-Oxide Semiconductor Field-Effect Transistor (MOSFET)

    \begin{itemize}

      \item Extremely high impedance looking into the gate

        \begin{itemize}
            
          \item $I_G\approx 0$

        \end{itemize}

      \item Body is often connected to the source or to ground (substrate)

      \item Cutoff Region

        \begin{itemize}

          \item $V_{GS}<V_{to}$

            \begin{itemize}

              \item $V_{to}$ is the threshold voltage (technology-dependent parameter: $.3[\si{\volt}]$ (new) - $2[\si{\volt}]$ (older))

            \end{itemize}

          \item $I_D=0$

        \end{itemize}

      \item Triode Region

        \begin{itemize}

          \item $V_{GS}>V_{to}$ and $V_{DS}<V_{GS}-V_{to}$

          \item Also called ``linear region''

            \begin{itemize}

              \item $I_D\propto V_{GS}$ when $V_{DS}$ is small

              \item Voltage-controlled resistance (between drain and source terminals)

            \end{itemize}

        \end{itemize}

      \item Saturation Region

        \begin{itemize}

          \item $V_{GS}>V_{to}$ and $V_{DS}>V_{GS}-V_{to}$

          \item $I_D=K(V_{GS}-V_{to})^2$ (desired mode for amplifiers)

        \end{itemize}

      \item Drain-Source Current:

        $$I_D=\left( \frac{W}{L} \right)\left( \frac{KP}{2} \right)\left[ 2(V_{GS}-V_{to})V_{DS}-V_{DS}^2 \right]\left[ 1+\lambda V_{DS} \right]$$

        \begin{itemize}

          \item Parameters:

            \begin{itemize}

              \item $L$ is the channel length, $W$ is the channel width

              \item $KP=\mu_nC_{ox}=\mu_n(\varepsilon_{ox}/t_{ox})$

                \begin{itemize}

                  \item $\mu_n$ is the mobility of the electrons in the channel

                  \item $C_{ox}$ is the oxide capacitance per unit area

                \end{itemize}

              \item $K=(W/L)(KP/2)$ has units of amp\'eres per square volt

              \item $\lambda$ is the channel length modulation parameter ($\lambda=0$ in many hand-based calculation estimates)

            \end{itemize}

        \end{itemize}

      \item Boundary between triode and saturation regions:

        $$I=\left( \frac{W}{L} \right)\left( \frac{KP}{2} \right)V_{DS}^2$$

    \end{itemize}

\end{itemize}

\end{document}

