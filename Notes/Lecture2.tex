%%%%%%%%%%%%%%%%%%%%%%%%%%%%%%%%%%%%%%%%%%%%%%%%%%%%%%%%%%%%%%%%%%%%%%%%%%%%%%%%%%%%%%%%%%%%%%%%%%%%%%%%%%%%%%%%%%%%%%%%%%%%%%%%%%%%%%%%%%%%%%%%%%%%%%%%%%%%%%%%%%%
% Written By Michael Brodskiy
% Class: Fundamentals of Electronics
% Professor: M. Onabajo
%%%%%%%%%%%%%%%%%%%%%%%%%%%%%%%%%%%%%%%%%%%%%%%%%%%%%%%%%%%%%%%%%%%%%%%%%%%%%%%%%%%%%%%%%%%%%%%%%%%%%%%%%%%%%%%%%%%%%%%%%%%%%%%%%%%%%%%%%%%%%%%%%%%%%%%%%%%%%%%%%%%

\include{Includes.tex}

\title{Lecture 2}
\date{\today}
\author{Michael Brodskiy\\ \small Professor: M. Onabajo}

\begin{document}

\maketitle

\begin{itemize}

  \item Current-Amplifier Model

    \begin{figure}[H]
      \centering
      \includegraphics[width=.7\textwidth]{Images/CAM.png}
      \caption{Reference Figure for Current-Amplifier Model}
      \label{fig:1}
    \end{figure}

    \begin{itemize}

      \item Parameters

        \begin{itemize}

          \item $i_i$ is the input current, which ideally comes from a current source

          \item $R_i$ and $R_o$ are the input and output resistances, respectively

          \item $A_{isc}$ is the short-circuit current gain

        \end{itemize}

      \item Current gain with load impedance at the output: $A_i=i_o/i_i$

    \end{itemize}

  \item Application of Th\'evenin to Norton transformation

    \begin{itemize}

      \item The connection of $R_o$ is changed, but the value remains the same

    \end{itemize}

  \item $A_{isc}=i_{osc}/i_i$ is obtained with a short-circuit at the output terminals

    \begin{itemize}

      \item where: $i_{osc}=A_{vo}v_i/R_o$ and $i_i=v_i/R_i$

      \item After substituting: $A_{isc}=A_{vo}(R_i/R_o)$

    \end{itemize}

\end{itemize}

\end{document}

