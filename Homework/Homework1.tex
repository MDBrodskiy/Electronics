%%%%%%%%%%%%%%%%%%%%%%%%%%%%%%%%%%%%%%%%%%%%%%%%%%%%%%%%%%%%%%%%%%%%%%%%%%%%%%%%%%%%%%%%%%%%%%%%%%%%%%%%%%%%%%%%%%%%%%%%%%%%%%%%%%%%%%%%%%%%%%%%%%%%%%%%%%%%%%%%%%%
% Written By Michael Brodskiy
% Class: Fundamentals of Electronics
% Professor: I. Salama
%%%%%%%%%%%%%%%%%%%%%%%%%%%%%%%%%%%%%%%%%%%%%%%%%%%%%%%%%%%%%%%%%%%%%%%%%%%%%%%%%%%%%%%%%%%%%%%%%%%%%%%%%%%%%%%%%%%%%%%%%%%%%%%%%%%%%%%%%%%%%%%%%%%%%%%%%%%%%%%%%%%

\documentclass[12pt]{article} 
\usepackage{alphalph}
\usepackage[utf8]{inputenc}
\usepackage[russian,english]{babel}
\usepackage{titling}
\usepackage{amsmath}
\usepackage{graphicx}
\usepackage{enumitem}
\usepackage{amssymb}
\usepackage[super]{nth}
\usepackage{everysel}
\usepackage{ragged2e}
\usepackage{geometry}
\usepackage{multicol}
\usepackage{fancyhdr}
\usepackage{cancel}
\usepackage{siunitx}
\usepackage{physics}
\usepackage{tikz}
\usepackage{mathdots}
\usepackage{yhmath}
\usepackage{cancel}
\usepackage{color}
\usepackage{array}
\usepackage{multirow}
\usepackage{gensymb}
\usepackage{tabularx}
\usepackage{extarrows}
\usepackage{booktabs}
\usepackage{lastpage}
\usetikzlibrary{fadings}
\usetikzlibrary{patterns}
\usetikzlibrary{shadows.blur}
\usetikzlibrary{shapes}

\geometry{top=1.0in,bottom=1.0in,left=1.0in,right=1.0in}
\newcommand{\subtitle}[1]{%
  \posttitle{%
    \par\end{center}
    \begin{center}\large#1\end{center}
    \vskip0.5em}%

}
\usepackage{hyperref}
\hypersetup{
colorlinks=true,
linkcolor=blue,
filecolor=magenta,      
urlcolor=blue,
citecolor=blue,
}


\title{Homework 1}
\date{\today}
\author{Michael Brodskiy\\ \small Professor: M. Onabajo}

\begin{document}

\maketitle

\begin{enumerate}

  \item

    \begin{center}
      We can set up and calculate:
    \end{center}
    $$v_i=v_s\frac{R_i}{R_s+R_i}=5\frac{10^6}{10^6+10^5}=4.545[\si{\volt}_{rms}]$$
    \begin{center}
      Given that the open-circuit voltage gain is unity, we may write:
    \end{center}
    $$v_o=4.545[\si{\volt}_{pp}]\frac{50}{100+50}$$
    $$\boxed{\therefore v_o=1.515[\si{\volt}_{rms}]}$$

    \begin{center}
      Power can then be determined using the $50[\si{\ohm}]$ load:
    \end{center}
    $$P_L=\frac{v_o^2}{R}=\frac{1.515^2}{50}$$
    $$\boxed{\therefore P_L=45.9[\si{\milli\watt}]}$$

    \begin{center}
      With a direct signal source connection, we can use voltage division to find:
    \end{center}
    $$v_L=v_s\frac{R_L}{R_L+R_s}=5\frac{50}{50+10^5}$$
    $$\boxed{\therefore v_L=2.5[\si{\milli\volt}]}$$

    \begin{center}
      The power can then be found using:
    \end{center}
    $$P_L=\frac{(2.5\cdot10^{-3})^2}{50}$$
    $$\boxed{\therefore P_L=.125[\si{\micro\watt}]}$$

    With the implementation of an amplifier, we see that the power is significantly increased. Similarly, the voltage delivered is also significantly increased. Thus, the use of an amplifier can greatly help with power delivery.

  \item

    \begin{itemize}

      \item For $A-B$ cascade:
        
        \begin{center}
          The input impedance can be found to be:
        \end{center}
        $$\boxed{R_i=R_{i1}=3[\si{\kilo\ohm}]}$$

        \begin{center}
          The output impedance can be found to be:
        \end{center}
        $$\boxed{R_o=R_{o2}=20[\si{\ohm}]}$$

        \begin{center}
          We may then proceed to find the gain using individual gains as steps:
        \end{center}
        $$A_{v1}=\frac{V_{o1}}{V_i}=\frac{100V_i\left[ \frac{10^6}{10^6+400} \right]}{V_i}$$
        $$A_{v1}=99.96=20\log(99.96)=39.97[\text{dB}]$$
        $$A_{v2}=\frac{V_{o2}}{V_{o1}}=\frac{500V_{o1}}{V_{o1}}$$
        $$A_{v2}=500=20\log(500)=53.979[\text{dB}]$$

        \begin{center}
          We then multiply to find the overall gain:
        \end{center}
        $$A_{voc}=A_{v1}A_{v2}=(500)(99.96)$$
        $$\boxed{A_{voc}=49.98\cdot10^3=93.976[\text{dB}]}$$

      \item For $B-A$ cascade:

        \begin{center}
          The input impedance can be found to be:
        \end{center}
        $$\boxed{R_i=R_{i2}=1[\si{\mega\ohm}]}$$

        \begin{center}
          The output impedance can be found to be:
        \end{center}
        $$\boxed{R_o=R_{o1}=400[\si{\ohm}]}$$

        \begin{center}
          We may then proceed to find the gain using individual gains as steps:
        \end{center}
        $$A_{v2}=\frac{V_{o2}}{V_i}=\frac{500V_i\left[ \frac{3000}{3000+20} \right]}{V_i}$$
        $$A_{v2}=496.69=20\log(496.69)=53.92[\text{dB}]$$
        $$A_{v1}=\frac{V_{o1}}{V_{o2}}=\frac{100V_{o2}}{V_{o2}}$$
        $$A_{v1}=100=20\log(100)=40[\text{dB}]$$

        \begin{center}
          We then multiply to find the overall gain:
        \end{center}
        $$A_{voc}=A_{v1}A_{v2}=(100)(496.69)$$
        $$\boxed{A_{voc}=49.669\cdot10^3=93.922[\text{dB}]}$$

    \end{itemize}

  \item

    \begin{center}
      We begin by finding $A_{vs}$, the voltage gain from source to output:
    \end{center}
    $$A_{vs}=\frac{V_o}{V_s}$$
    $$i_i=\frac{V_s}{R_{i1}+R_s}=\frac{V_s}{55000}$$
    $$V_{moc}=\frac{10}{55}V_s$$
    $$V_{i2}=\frac{V_{moc}R_{i2}}{R_{i2}+R_{o1}}=\left( \frac{10}{55}V_s \right)\left( \frac{10^6}{10^6+200} \right)$$
    $$V_{i2}=.1818V_s$$
    $$i_{msc}=(.02)(.1818V_s)=(3.6356\cdot10^{-3})V_s$$
    $$|V_o|=(i_{msc})\frac{R_{o2}R_L}{R_{o2}+R_L}=(3.6356\cdot10^{-3})V_s\left( \frac{(10^5)(10^3)}{10^5+10^3} \right)$$
    $$|V_o|=3.5996V_s[\si{\volt}]$$

    \begin{center}
      First, because the current is travelling in a direction opposing $i_o$, we flip the sign, and then find the gain:
    \end{center}
    $$V_o=-3.5996V_s[\si{\volt}]$$
    $$\boxed{A_{vs}=-\frac{3.5996V_s}{V_s}=-3.5996}$$

    \begin{center}
      Using $V_{i1}$, from above, we can find the loaded voltage gain:
    \end{center}
    $$\boxed{A_v=-\frac{3.5996V_s}{(50/55)V_s}=-3.9596}$$

    \begin{center}
      Using $i_{i}$, from above, we can find the overall current gain:
    \end{center}
    $$i_o=\frac{V_o}{1000}=-.0035996V_s$$
    $$\boxed{A_i=-\frac{.0035996V_s}{(1/55000)V_s}=-197.98}$$

    \begin{center}
      Finally, we get the power gain:
    \end{center}
    $$A_p=(3.9596)(197.98)=783.91$$

  \item For the operational amplifier:

    \begin{enumerate}

      \item

        \begin{center}
          We can find that the voltage gain is:
        \end{center}
        $$\boxed{A_{v}=\frac{7.5}{.02}=375=51.48[\si{\volt}\text{dB}]}$$

        \begin{center}
          We can find that the current gain is:
        \end{center}
        $$\boxed{A_{i}=\frac{(7.5/.5)\cdot10^{-3}}{10^{-6}}=15000=83.522[\text{dB}]}$$

        \begin{center}
          Combining the two together, the power gain is:
        \end{center}
        $$\boxed{A_{p}=(375)(15000)=5.625\cdot10^6=67.501[\text{dB}]}$$

        \begin{center}
          Finally, the input resistance is defined as:
        \end{center}
        $$\boxed{R_{i}=\frac{.02}{10^6}=20[\si{\kilo\ohm}]}$$

      \item 

        \begin{center}
          The power delivered to the amplifier may be found as:
        \end{center}
        $$\boxed{P_s=2(12)(.01)=.24[\si{\watt}]}$$

        \begin{center}
          To find the efficiency, we must first find the output power:
        \end{center}
        $$P_o=\frac{1}{1000}(7.5)^2=56.25[\si{\milli\watt}]$$

        \begin{center}
          Thus, we find the efficiency to be:
        \end{center}
        $$\eta=\frac{P_o}{P_s}\cdot100=\frac{56.25}{2.40}$$
        $$\boxed{\eta=23.44\%}$$

      \item 

        \begin{center}
          The max voltage may be calculated as follows:
        \end{center}
        $$V_{max}=\frac{V_{dc}}{A_v}=\frac{12}{375}$$
        $$\boxed{V_{max}=32[\si{\milli\volt}]}$$

    \end{enumerate}

  \item

    \begin{enumerate}

      \item We begin by finding the impedance of the capacitor:

        $$C_L=\frac{1}{j\omega C}=\frac{10^7}{s}$$

        \begin{center}
          From here, we begin to solve the circuit:
        \end{center}
        $$V_i=\frac{150V_s}{200}=.75V_s$$
        $$i_{i}=.075V_s$$

        \begin{center}
          Here, we solve for the equivalent impedance:
        \end{center}
        $$R_{eq}=\left( \frac{(15000)(5000)}{20000} \right)=3750[\si{\ohm}]$$
        $$Z_{eq}=\left( \frac{(3750)(10^7/s)}{3750+(10^7/s)} \right)[\si{\ohm}]$$
        $$Z_{eq}=\frac{10^7}{s+2666.67}[\si{\ohm}]$$

        \begin{center}
          We now multiply by the current to find $V_o$:
        \end{center}
        $$Z_{eq}i_i=\left(\frac{10^7}{s+2666.67}\right)\left(  .075V_s\right)$$
        $$V_o=\frac{75\cdot10^4V_s}{s+2666.67}$$

        \begin{center}
          This finally gives us:
        \end{center}
        $$A_{vs}(s)=\frac{75\cdot10^4}{s+2666.67}$$
        $$\boxed{A_{vs}(j\omega)=\frac{75\cdot10^4}{j\omega+2666.67}}$$

      \item The 3dB frequency may be found using the magnitude of the transfer function:

        $$|A_{vs}(j\omega)|=\frac{1}{\sqrt{2}}$$
        $$|\frac{750000}{\sqrt{\omega^2+2666.67^2}}|=\frac{1}{\sqrt{2}}$$
        $$\omega^2+2666.67^2=1.125\cdot10^{12}$$
        $$\omega^2=1.125\cdot10^{12}$$
        $$\omega\approx1.0607\cdot10^{6}\left[ \frac{\text{rad}}{\si{\second}} \right]$$

        \begin{center}
          Finally, this yields:
        \end{center}
        $$f=\frac{\omega}{2\pi}$$
        $$\boxed{f_c=1.688\cdot10^5[\si{\hertz}]}$$

      \item The gain-bandwidth is found as follows:

        $$G_{BW}=\frac{750000}{\sqrt{(1.0607\cdot10^6)^2+2666.67^2}}(1.0607\cdot10^6)$$
        $$\boxed{G_{BW}=749998}$$

    \end{enumerate}

  \item The voltage gain for the first operational amplifer can be found according to the zero current flow into the terminals is zero:

    $$A_1=\frac{v_{o1}}{v_i}=-\frac{4R}{R}$$
    $$\boxed{A_1=-4}$$

    \begin{center}
      We repeat the same process for the second op-amp:
    \end{center}
    $$A_2=\frac{v_{o2}}{v_i}=-\frac{4R}{R}$$
    $$\boxed{A_2=-4}$$

    \begin{center}
      Thus, thanks to the summing-point constraint, we see both voltages are inverted, and the gains are 4.
    \end{center}

\end{enumerate}

\end{document}

