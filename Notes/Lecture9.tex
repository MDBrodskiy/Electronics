%%%%%%%%%%%%%%%%%%%%%%%%%%%%%%%%%%%%%%%%%%%%%%%%%%%%%%%%%%%%%%%%%%%%%%%%%%%%%%%%%%%%%%%%%%%%%%%%%%%%%%%%%%%%%%%%%%%%%%%%%%%%%%%%%%%%%%%%%%%%%%%%%%%%%%%%%%%%%%%%%%%
% Written By Michael Brodskiy
% Class: Fundamentals of Electronics
% Professor: M. Onabajo
%%%%%%%%%%%%%%%%%%%%%%%%%%%%%%%%%%%%%%%%%%%%%%%%%%%%%%%%%%%%%%%%%%%%%%%%%%%%%%%%%%%%%%%%%%%%%%%%%%%%%%%%%%%%%%%%%%%%%%%%%%%%%%%%%%%%%%%%%%%%%%%%%%%%%%%%%%%%%%%%%%%

\include{Includes.tex}

\title{Lecture 9}
\date{\today}
\author{Michael Brodskiy\\ \small Professor: M. Onabajo}

\begin{document}

\maketitle

\begin{itemize}

  \item Rectifiers

    \begin{itemize}

      \item Simple Half-Wave Recitifers

        \begin{itemize}

          \item Used for AC-to-DC conversion

          \item Circuit representation when the diode is forward biased

            \begin{itemize}

              \item $V_o=V_i$

              \item $I_D=V_O/R_L$

            \end{itemize}

          \item Average output voltage: $V_{oAVG}=V_{oPK}/\pi$

          \item RMS output voltage: $V_{oRMS}=V_{oPK}/2$

        \end{itemize}

      \item Half-Wave Rectifier with Smoothing Capacitor

        \begin{itemize}

          \item Addition of a capacitor in parallel with the load resistor

            \begin{itemize}

              \item Formation of a first-order low-pass filter

              \item Impedance of $R_L$ and $C$ in parallel

                $$Z_P=\frac{1}{(1/R_L)+sC}=\frac{R_L}{1+j(\omega/\omega_B)},\quad\text{ where }\omega_B=(1/RC)$$

            \end{itemize}

          \item The capacitor charges to $V_i$ when the diode is on $(I_D=I_C+I_L)$

            \begin{itemize}

              \item $I_C>0$ when $V_o=V_c<V_i$, $I_C=0$ when $V_O=V_C=V_i$ (ideal model)

            \end{itemize}

          \item The capacitor holds (stores the output voltage when the diode is off)

              \begin{itemize}

                \item $V_o$ reduces gradually due to discharge of the capacitor: $I_C=-I_L\new 0$

              \end{itemize}

            \item Average output voltage: $V_{oAVG}\approx V_{iPK}- (V_r/2)$

              \begin{itemize}

                \item $V_r$ is the ripple voltage (design starting point: only for the half-wave rectifier)

                \item Goal for a given input signal period ($T$): minimum $V_r\to$ maximum $V_o$

              \end{itemize}

        \end{itemize}

    \end{itemize}

  \item Electrostatic Discharge (ESD) Protection

    \begin{itemize}
        
      \item ESD event

        \begin{itemize}

          \item Rapid discharge of charge between two bodies at different potentials

          \item High voltage build-up $\to$ current flow (can destroy internal circuits)

          \item Occurs during the manufacturing and lifetime of a chip (pins come into contact with equipment and sometimes people)

        \end{itemize}

      \item Simplified ESD protection concept with diodes

    \end{itemize}

  \item Transformers

    \begin{itemize}

      \item Convert voltage based on a turns ratio, $N_p:N_s$, or simplified $n:1$, where:

        $$V_s=\frac{V_p}{n}$$

    \end{itemize}

  \item Full-Wave Rectifiers

    \begin{itemize}

      \item Begins with a transformer, which isolates AC input to bridge from ground

      \item Proceeds to a square arrangement of diodes, with one node grounded (current path for positive half-cycle)

      \item Proceeds to a load resistance connected to ground

    \end{itemize}

\end{itemize}

\end{document}

