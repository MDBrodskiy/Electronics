%%%%%%%%%%%%%%%%%%%%%%%%%%%%%%%%%%%%%%%%%%%%%%%%%%%%%%%%%%%%%%%%%%%%%%%%%%%%%%%%%%%%%%%%%%%%%%%%%%%%%%%%%%%%%%%%%%%%%%%%%%%%%%%%%%%%%%%%%%%%%%%%%%%%%%%%%%%%%%%%%%%
% Written By Michael Brodskiy
% Class: Fundamentals of Electronics
% Professor: M. Onabajo
%%%%%%%%%%%%%%%%%%%%%%%%%%%%%%%%%%%%%%%%%%%%%%%%%%%%%%%%%%%%%%%%%%%%%%%%%%%%%%%%%%%%%%%%%%%%%%%%%%%%%%%%%%%%%%%%%%%%%%%%%%%%%%%%%%%%%%%%%%%%%%%%%%%%%%%%%%%%%%%%%%%

\documentclass[12pt]{article} 
\usepackage{alphalph}
\usepackage[utf8]{inputenc}
\usepackage[russian,english]{babel}
\usepackage{titling}
\usepackage{amsmath}
\usepackage{graphicx}
\usepackage{enumitem}
\usepackage{amssymb}
\usepackage[super]{nth}
\usepackage{everysel}
\usepackage{ragged2e}
\usepackage{geometry}
\usepackage{multicol}
\usepackage{fancyhdr}
\usepackage{cancel}
\usepackage{siunitx}
\usepackage{physics}
\usepackage{tikz}
\usepackage{mathdots}
\usepackage{yhmath}
\usepackage{cancel}
\usepackage{color}
\usepackage{array}
\usepackage{multirow}
\usepackage{gensymb}
\usepackage{tabularx}
\usepackage{extarrows}
\usepackage{booktabs}
\usepackage{lastpage}
\usetikzlibrary{fadings}
\usetikzlibrary{patterns}
\usetikzlibrary{shadows.blur}
\usetikzlibrary{shapes}

\geometry{top=1.0in,bottom=1.0in,left=1.0in,right=1.0in}
\newcommand{\subtitle}[1]{%
  \posttitle{%
    \par\end{center}
    \begin{center}\large#1\end{center}
    \vskip0.5em}%

}
\usepackage{hyperref}
\hypersetup{
colorlinks=true,
linkcolor=blue,
filecolor=magenta,      
urlcolor=blue,
citecolor=blue,
}


\title{Homework 10}
\date{\today}
\author{Michael Brodskiy\\ \small Professor: M. Onabajo}

\begin{document}

\maketitle

\begin{enumerate}

  \item

    \begin{enumerate}

      \item Since we see that $V_{GS}<V_{to}$, the transistor is operating in the cutoff region. In the cutoff region, the drain current is $\boxed{I_D=0}$

      \item Since $V_{DS}\leq V_{DS}-V_{to}$ and $V_{GS}\geq V_{to}$, the transistor is operating in the linear (triode) region. This gives us the drain current as:

        $$I_D=\left( \frac{W}{L} \right)\left( \frac{KP}{2} \right)\left[ 2(V_{GS}-V_{to})V_{DS}-V_{DS}^2 \right]\left[ 1+\lambda V_{DS} \right]$$
        $$I_D=\left( 10 \right)\left( 25\cdot10^{-6} \right)\left[ 2(-1)(5)-(5)^2 \right]\left[ 1+(0)(5) \right]$$
        $$\boxed{I_D=-8.75[\si{\milli\ampere}]}$$

      \item Since $V_{DS}\geq V_{GS}-V_{to}$ and $V_{GS}\geq V_{to}$, the transistor is operating in the saturation region. The drain current becomes:

        $$I_{D}=KP\left( \frac{W}{L} \right)(V_{G}-V_{S}-V_t)^2$$
        $$I_{D}=\left( 50\cdot10^{-6} \right)\left( 10 \right)(3-1)^2$$
        $$\boxed{I_{D}=2[\si{\milli\ampere}]}$$

      \item Since $V_{DS}\geq V_{GS}-V_{to}$ and $V_{GS}\geq V_{to}$, the transistor is operating in the saturation region. The drain current becomes:

        $$I_{D}=KP\left( \frac{W}{L} \right)(V_{G}-V_{S}-V_t)^2$$
        $$I_{D}=\left( 50\cdot10^{-6} \right)\left( 10 \right)(5-1)^2$$
        $$\boxed{I_{D}=8[\si{\milli\ampere}]}$$


    \end{enumerate}

  \item First, we know that $I_G=0$ since the input impedance of the MOSFET is high. In this manner, we may write:

    $$I_{DQ}=I_{SQ}$$

    Using KVL, we may obtain:

    $$V_{DD}=V_{GS}+I_{DQ}R_S$$
    $$V_{GS}=15-3000I_{DQ}$$

    Assuming the MOSFET is operating in the saturated region, we may write:

    $$I_{DQ}=K(V_{GS}-V_{to})^2$$
    $$I_{DQ}=.25(15-3000I_{DQ}-1)^2$$
    $$I_{DQ}=.25(14-3000I_{DQ})^2$$
    $$I_{DQ}=2250I_{DQ}^2-21I_{DQ}+.049$$
    $$0=2250I_{DQ}^2-22I_{DQ}+.049$$

    Solving the equation, we obtain:

    $$I_{DQ}=4.889\cdot10^{-3}\pm1.4572\cdot10^{-4}$$
    $$\boxed{I_{DQ}=6.3461,\,3.4317[\si{\milli\ampere}]}$$

    We now check the voltage in both cases. Let us use the first value to find the gate-to-source voltage:

    $$V_{GS1}=15-(3000)(I_{DQ})$$
    $$V_{GS1}=15-(3)(6.3461)$$
    $$\boxed{V_{GS1}=-4.0383[\si{\volt}]}$$

    We may observe that, in this case, the transistor is off. Now, we use the second value:

    $$V_{GS2}=15-(3000)(I_{DQ})$$
    $$V_{GS2}=15-(3)(3.4317)$$
    $$\boxed{V_{GS2}=4.7049[\si{\volt}]}$$

    We see that the transistor is on only for the second value. Thus, we proceed with the second drain current value. This gives us:

    $$V_{DD}-I_{DQ}(R_D)-V_{DSQ}-I_{DQ}(R_S)+V_{DD}=0$$
    $$30-I_{DQ}(R_D)-I_{DQ}(R_S)=V_{DSQ}$$

    We can solve using our known values:

    $$V_{DSQ}=30-3.4317(4)$$
    $$\boxed{V_{DSQ}=16.273[\si{\volt}]}$$

    We may observe that both $V_{GS}>V_{to}$ and $V_{DSQ}\geq V_{GS}-V_{to}$ are true, meaning that our saturation assumption was valid. As such, we have found our values for the given transistor.

  \item First and foremost, we know that $M_1$ is in saturation since $V_{D1}>(V_{G1}-V_t)$. As such, we may write:

    $$I_{D1}=KP\left( \frac{W}{L} \right)_1\left( V_{G1}-V_{S1}-V_t \right)^2$$
    $$I_{D1}=30\cdot10^{-6}\left( 40 \right)\left( 5-3-1 \right)^2$$
    $$\boxed{I_{D1}=1.2[\si{\milli\ampere}]}$$

    From here, since we know $I_{D1}=I_{D2}$, we may write:

    $$I_{D1}=KP\left( \frac{W}{L} \right)_2(V_{G2}-V_{S2}-V_t)^2$$
    $$1.2\cdot10^{-3}=30\cdot10^{-6}\left( \frac{W}{L} \right)_2(3-0-1)^2$$
    $$1.2\cdot10^{-3}=120\cdot10^{-6}\left( \frac{W}{L} \right)_2$$
    $$\boxed{\left( \frac{W}{L} \right)_2=10}$$

\end{enumerate}

\end{document}

