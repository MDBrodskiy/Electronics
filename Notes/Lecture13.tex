%%%%%%%%%%%%%%%%%%%%%%%%%%%%%%%%%%%%%%%%%%%%%%%%%%%%%%%%%%%%%%%%%%%%%%%%%%%%%%%%%%%%%%%%%%%%%%%%%%%%%%%%%%%%%%%%%%%%%%%%%%%%%%%%%%%%%%%%%%%%%%%%%%%%%%%%%%%%%%%%%%%
% Written By Michael Brodskiy
% Class: Fundamentals of Electronics
% Professor: M. Onabajo
%%%%%%%%%%%%%%%%%%%%%%%%%%%%%%%%%%%%%%%%%%%%%%%%%%%%%%%%%%%%%%%%%%%%%%%%%%%%%%%%%%%%%%%%%%%%%%%%%%%%%%%%%%%%%%%%%%%%%%%%%%%%%%%%%%%%%%%%%%%%%%%%%%%%%%%%%%%%%%%%%%%

\include{Includes.tex}

\title{Lecture 13}
\date{\today}
\author{Michael Brodskiy\\ \small Professor: M. Onabajo}

\begin{document}

\maketitle

\begin{itemize}

  \item General Amplifier Analysis Procedure

    \begin{itemize}

      \item DC bias circuit analysis/design

        \begin{itemize}

          \item Consider only DC sources (remove AC sources)

          \item Ensure operation in the active region

          \item Take desired Q-point parameters into account ($g_m$, $r_\pi$, etc.)

        \end{itemize}

      \item AC Analysis

        \begin{itemize}

          \item Draw the AC equivalent circuit with the appropriate BJT model

          \item Consider only AC sources (remove DC sources)

          \item Midband (medium-frequency AC analysis):

              \begin{itemize}

                \item Capacitor $\to$ short-circuit, inductor $\to$ open-circuit

              \end{itemize}

            \item Frequency-dependent analysis

              \begin{itemize}

                \item Capacitor and inductor go to their frequency-dependent impedances

              \end{itemize}

        \end{itemize}

      \item Removal of Sources

        \begin{itemize}

          \item Voltage source $\to$ replace with a short-circuit

          \item Current source $\to$ replace with an open-circuit

        \end{itemize}

    \end{itemize}

  \item The Early Effect (BJT Output Resistance)

    \begin{itemize}

      \item As $V_C$ increases, the depletion width of the B-C junction widens

    \end{itemize}

  \item The Darlington Pair Configuration

    \begin{itemize}

      \item Two transistors, cascaded with the second's base connected to the first's emitter

      \item $\beta_{eff}=\beta_1+\beta_2+\beta_1\beta_2\approx \beta_1\beta_2$

      \item $V_{BEeff}=V_{BE1}+_{BE2}\approx 2V_{BE1}$

      \item For active mode: $V_{CE}>V_{BE2}+V_{CE1(min)}$

    \end{itemize}

  \item Current Sources

    \begin{itemize}

      \item Ideal

        \begin{itemize}

          \item An ideal source will tolerate any voltage ($V_z$) across its terminals

          \item Infinite output resistance

        \end{itemize}

      \item Practical

        \begin{itemize}

          \item A minimum compliance voltage must be maintained to guarantee a certain output current: $V_z>V_{comp}$

          \item Finite output resistance

        \end{itemize}

    \end{itemize}

  \item Current Mirror concept

    \begin{itemize}

      \item Often, a ``golden current source'' is replicated at multiple locations on the chip or printed circuit board (PCB)

      \item Integrated circuit applications

        \begin{itemize}

          \item $I_{REF}$ can be on the chip or outside of the chip

          \item Accurate reference ($I_{REF}$) generation is costly (expensive components or requirement for a lot of chip area)

          \item The use of current mirrors is more efficient than generating multiple reference currents

        \end{itemize}

    \end{itemize}

  \item BJT as a Current Source

    \begin{itemize}

      \item Requirements for a reliable collector current:

        \begin{itemize}

          \item A fixed $V_{BE}$ must be generated

          \item The BJT should operate in the active region ($V_z>V_{CEmin}$)

        \end{itemize}

      \item A Diode-Connected BJT

        \begin{itemize}

          \item Connect a reference current source to the collector, then short the collector to base

          \item This ensures $V_C=V_B$, and forces the BJT to remain in the active region. The emitter generates $Q_{ref}$ (``diode-connected'')

          \item The reference voltage generation:

            $$I_C=I_S\left[ e^{V_{BE}/V_T}-1 \right]\approx I_se^{V_{BE}/V_T}$$
            $$V_1=V_{BE}\approx V_T\ln\left( \frac{I_C}{I_S} \right)$$

        \end{itemize}

    \end{itemize}

  \item A Simple Current Mirror

    \begin{itemize}

      \item The diode-connected $Q_{ref}$ produces a reference voltage at the base of the diode-connect BJT

      \item Connect another BJT's base to the base of the diode-connected BJT

      \item $V_{em}=0[\si{\volt}]$ and $V_{ba}=V_x$ are identical for both BJTs

        \begin{itemize}

          \item Forces $I_{mirror}$ (the current into the collector of the second BJT) to be equal to the reference current $I_{ref}$ (ignoring base currents) if $Q_1=Q_{ref}$

          \item With different BJT characteristics:

            $$I_{mirror}\approx \frac{A_1}{A_{ref}}I_{ref}=\frac{I_{S1}}{I_{Sref}}I_{ref}$$

        \end{itemize}

    \end{itemize}

  \item Resistor-Transistor Logic (RTL)

    \begin{itemize}

      \item Gives good insights into the concept of using analog transistors for digital signal processing

      \item Logic states: high (``1'') or low (``0'')

      \item RTL logic levels: $V_{CC}$ or $\approx .2-.3[\si{\volt}]$

      \item Not popular anymore

    \end{itemize}

\end{itemize}

\end{document}

