%%%%%%%%%%%%%%%%%%%%%%%%%%%%%%%%%%%%%%%%%%%%%%%%%%%%%%%%%%%%%%%%%%%%%%%%%%%%%%%%%%%%%%%%%%%%%%%%%%%%%%%%%%%%%%%%%%%%%%%%%%%%%%%%%%%%%%%%%%%%%%%%%%%%%%%%%%%%%%%%%%%
% Written By Michael Brodskiy
% Class: Fundamentals of Electronics
% Professor: M. Onabajo
%%%%%%%%%%%%%%%%%%%%%%%%%%%%%%%%%%%%%%%%%%%%%%%%%%%%%%%%%%%%%%%%%%%%%%%%%%%%%%%%%%%%%%%%%%%%%%%%%%%%%%%%%%%%%%%%%%%%%%%%%%%%%%%%%%%%%%%%%%%%%%%%%%%%%%%%%%%%%%%%%%%

\documentclass[12pt]{article} 
\usepackage{alphalph}
\usepackage[utf8]{inputenc}
\usepackage[russian,english]{babel}
\usepackage{titling}
\usepackage{amsmath}
\usepackage{graphicx}
\usepackage{enumitem}
\usepackage{amssymb}
\usepackage[super]{nth}
\usepackage{everysel}
\usepackage{ragged2e}
\usepackage{geometry}
\usepackage{multicol}
\usepackage{fancyhdr}
\usepackage{cancel}
\usepackage{siunitx}
\usepackage{physics}
\usepackage{tikz}
\usepackage{mathdots}
\usepackage{yhmath}
\usepackage{cancel}
\usepackage{color}
\usepackage{array}
\usepackage{multirow}
\usepackage{gensymb}
\usepackage{tabularx}
\usepackage{extarrows}
\usepackage{booktabs}
\usepackage{lastpage}
\usetikzlibrary{fadings}
\usetikzlibrary{patterns}
\usetikzlibrary{shadows.blur}
\usetikzlibrary{shapes}

\geometry{top=1.0in,bottom=1.0in,left=1.0in,right=1.0in}
\newcommand{\subtitle}[1]{%
  \posttitle{%
    \par\end{center}
    \begin{center}\large#1\end{center}
    \vskip0.5em}%

}
\usepackage{hyperref}
\hypersetup{
colorlinks=true,
linkcolor=blue,
filecolor=magenta,      
urlcolor=blue,
citecolor=blue,
}


\title{Lecture 11}
\date{\today}
\author{Michael Brodskiy\\ \small Professor: M. Onabajo}

\begin{document}

\maketitle

\begin{itemize}

  \item npn Bipolar Junction Transistors

    \begin{itemize}

      \item The collector ``collects'' electrons, and causes current to flow through the emitter

    \end{itemize}
    
  \item npn Structure without Bias

    \begin{itemize}

      \item At zero bias ($V_{be}=V_{bc}=0$), neither electrons nor holes can overcome this built-in voltage barrier of $\approx.7[\si{\volt}]$ (for Si)

        \begin{itemize}

          \item $I_B=I_C=0$ (cutoff)

        \end{itemize}

    \end{itemize}

  \item npn Structure with Forward-Biased EBJ

    \begin{itemize}

      \item When ($V_{be}=.65[\si{\volt}]$, $V_c>V_b$), electrons and holes can overcome the built-in voltage barrier between the base and emitter

        \begin{itemize}

          \item $I_b>0$ and $I_e>I_b$ (due to $n^+$ emitter doping)

        \end{itemize}

      \item If the base region is very thin, the electrons injected by the emitter are collected by the positive voltage applied at $V_c$

        \begin{itemize}

          \item $I_c\approx I_E >> I_B$ (active region)

        \end{itemize}

      \item If the base region is too thick, many electrons injected at the emitter are lost by recombining with holes in the base before the voltage applied at $V_c$ can collect them

        \begin{itemize}

          \item $I_c<I_E$ (active region with low $\alpha$ and $\beta\to$ low gain)

        \end{itemize}

    \end{itemize}

  \item Achievement of high $\beta$ during Fabrication

    \begin{itemize}

      \item Thin base region

        \begin{itemize}

          \item Increases the collection efficiency for injected electrons

          \item Reduces the chance of electron recombination in the base

        \end{itemize}

      \item Heavily-doped emitter

        \begin{itemize}

          \item $I_E/I_B\propto n (\text{emitter})/ p (\text{base})\propto \beta$

        \end{itemize}

      \item Doping concentrations are difficult to control precisely

        \begin{itemize}

          \item Current gain is not uniform among BJTs (exception: when the BJTs are all fabricated on the same integrated circuit $\to$ small variations)

        \end{itemize}

    \end{itemize}

  \item The Early Effect

    \begin{itemize}

      \item As $V_c$ increases, the depletion width of the B-C junction widens

        \begin{itemize}

          \item Base width becomes more narrow

          \item Increased collection efficiency

          \item Finally, $I_c/I_b$ increases (higher $\beta$)

        \end{itemize}

    \end{itemize}

\end{itemize}

\end{document}

