%%%%%%%%%%%%%%%%%%%%%%%%%%%%%%%%%%%%%%%%%%%%%%%%%%%%%%%%%%%%%%%%%%%%%%%%%%%%%%%%%%%%%%%%%%%%%%%%%%%%%%%%%%%%%%%%%%%%%%%%%%%%%%%%%%%%%%%%%%%%%%%%%%%%%%%%%%%%%%%%%%%
% Written By Michael Brodskiy
% Class: Fundamentals of Electronics
% Professor: M. Onabajo
%%%%%%%%%%%%%%%%%%%%%%%%%%%%%%%%%%%%%%%%%%%%%%%%%%%%%%%%%%%%%%%%%%%%%%%%%%%%%%%%%%%%%%%%%%%%%%%%%%%%%%%%%%%%%%%%%%%%%%%%%%%%%%%%%%%%%%%%%%%%%%%%%%%%%%%%%%%%%%%%%%%

\documentclass[12pt]{article} 
\usepackage{alphalph}
\usepackage[utf8]{inputenc}
\usepackage[russian,english]{babel}
\usepackage{titling}
\usepackage{amsmath}
\usepackage{graphicx}
\usepackage{enumitem}
\usepackage{amssymb}
\usepackage[super]{nth}
\usepackage{everysel}
\usepackage{ragged2e}
\usepackage{geometry}
\usepackage{multicol}
\usepackage{fancyhdr}
\usepackage{cancel}
\usepackage{siunitx}
\usepackage{physics}
\usepackage{tikz}
\usepackage{mathdots}
\usepackage{yhmath}
\usepackage{cancel}
\usepackage{color}
\usepackage{array}
\usepackage{multirow}
\usepackage{gensymb}
\usepackage{tabularx}
\usepackage{extarrows}
\usepackage{booktabs}
\usepackage{lastpage}
\usetikzlibrary{fadings}
\usetikzlibrary{patterns}
\usetikzlibrary{shadows.blur}
\usetikzlibrary{shapes}

\geometry{top=1.0in,bottom=1.0in,left=1.0in,right=1.0in}
\newcommand{\subtitle}[1]{%
  \posttitle{%
    \par\end{center}
    \begin{center}\large#1\end{center}
    \vskip0.5em}%

}
\usepackage{hyperref}
\hypersetup{
colorlinks=true,
linkcolor=blue,
filecolor=magenta,      
urlcolor=blue,
citecolor=blue,
}


\title{Lecture 8}
\date{\today}
\author{Michael Brodskiy\\ \small Professor: M. Onabajo}

\begin{document}

\maketitle

\begin{itemize}

  \item PN-Junction Diodes

    \begin{itemize}

      \item Shockley Equation (``Diode Equation'')

        \begin{itemize}

          \item More realistic model for the I-V characteristics (for FB and RB regions)

          \item Based on semiconductor physics

            $$i_D=I_se^{V_D/(nV_T)}+1$$

        \end{itemize}

      \item $I_s$ is the saturation current ($10^{-6}$ to $10^{-18}[\si{\ampere}]$)

      \item $n$ = diode ideality factor, also called emissions coefficient, and can range from 1 to 2

      \item $V_T=(kT)/q$ is the thermal voltage ($\approx26[\si{\milli\volt}]$ at $T=300[\si{\kelvin}]$ room temperature)

        \begin{itemize}

          \item $k$ = Boltzmann's constant ($1.38\cdot10^{-23}[\si{\joule}/\si{\kelvin}]$), $q$ = electron charge ($1.6\cdot10^{-19}[\si{\coulomb}]$)

        \end{itemize}

    \end{itemize}

  \item Temperature Dependence

    \begin{itemize}

      \item At a constant current, the voltage drop decreases approximately $2[\si{\milli\volt}]$ for every $1[\si{\celsius}]$ increase in temperature

    \end{itemize}

  \item Solving Circuits using the Junction Diode Model

    \begin{itemize}

      \item Iterative Approach

        \begin{itemize}

          \item Pro: accurate hand calculations

          \item Con: tedious (time-consuming)

        \end{itemize}

      \item Graphical Approach

        \begin{itemize}

          \item Pro: fast

          \item Con: inaccurate (unless done numerically with a computer program)

        \end{itemize}

      \item Simulation

        \begin{itemize}

          \item Pro: most accurate

          \item Con: limited insights into the trade-offs

        \end{itemize}

    \end{itemize}

  \item Constant Voltage Drop (CVD) Model

    \begin{itemize}

      \item Diode approximation with an open-circuit (RB) or with a DC voltage source of $V_{dc}$ ($\approx .7[\si{\volt}]$) to model the forward voltage drop (FB)

      \item With the Resistance Model

        \begin{itemize}

          \item The diode is modeled with an open-circuit (RB), or with a DC voltage source and series resistance (FB), where $V_{do}\approx .7[\si{\volt}]$

          \item Dynamic resistance: $r_d=(nV_T)/I_{DQ}$

            \begin{itemize}

              \item Approximated around the operating point $Q$ (quiescent point)

              \item A small-signal parameter (represents the diode's resistance associated with small changes of $i_d$ and $V_d$)

            \end{itemize}

        \end{itemize}

    \end{itemize}

  \item Zener Diode Modeling

    \begin{itemize}

      \item Forward Bias (FB): $I_D>0$

      \item Reverse Bias (RB): $-V_{zo}<V_d<V_{do}$

      \item Breakdown (BD): $I_z=-I_D>0,\, V_D=-V_{zo}-I_zr_z$

      \item Voltage Regulation:

        \begin{itemize}

          \item Voltage regulator quality indicators:

            \begin{itemize}

              \item Source regulation: $\dfrac{\Delta V_{load}}{\Delta V_{SS}}\cdot100\% $

              \item Load regulation: $\dfrac{V_{NOload}-V_{FULLload}}{V_{FULLload}}\cdot100\% $

            \end{itemize}

        \end{itemize}

    \end{itemize}

\end{itemize}

\end{document}

