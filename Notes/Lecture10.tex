%%%%%%%%%%%%%%%%%%%%%%%%%%%%%%%%%%%%%%%%%%%%%%%%%%%%%%%%%%%%%%%%%%%%%%%%%%%%%%%%%%%%%%%%%%%%%%%%%%%%%%%%%%%%%%%%%%%%%%%%%%%%%%%%%%%%%%%%%%%%%%%%%%%%%%%%%%%%%%%%%%%
% Written By Michael Brodskiy
% Class: Fundamentals of Electronics
% Professor: M. Onabajo
%%%%%%%%%%%%%%%%%%%%%%%%%%%%%%%%%%%%%%%%%%%%%%%%%%%%%%%%%%%%%%%%%%%%%%%%%%%%%%%%%%%%%%%%%%%%%%%%%%%%%%%%%%%%%%%%%%%%%%%%%%%%%%%%%%%%%%%%%%%%%%%%%%%%%%%%%%%%%%%%%%%

\documentclass[12pt]{article} 
\usepackage{alphalph}
\usepackage[utf8]{inputenc}
\usepackage[russian,english]{babel}
\usepackage{titling}
\usepackage{amsmath}
\usepackage{graphicx}
\usepackage{enumitem}
\usepackage{amssymb}
\usepackage[super]{nth}
\usepackage{everysel}
\usepackage{ragged2e}
\usepackage{geometry}
\usepackage{multicol}
\usepackage{fancyhdr}
\usepackage{cancel}
\usepackage{siunitx}
\usepackage{physics}
\usepackage{tikz}
\usepackage{mathdots}
\usepackage{yhmath}
\usepackage{cancel}
\usepackage{color}
\usepackage{array}
\usepackage{multirow}
\usepackage{gensymb}
\usepackage{tabularx}
\usepackage{extarrows}
\usepackage{booktabs}
\usepackage{lastpage}
\usetikzlibrary{fadings}
\usetikzlibrary{patterns}
\usetikzlibrary{shadows.blur}
\usetikzlibrary{shapes}

\geometry{top=1.0in,bottom=1.0in,left=1.0in,right=1.0in}
\newcommand{\subtitle}[1]{%
  \posttitle{%
    \par\end{center}
    \begin{center}\large#1\end{center}
    \vskip0.5em}%

}
\usepackage{hyperref}
\hypersetup{
colorlinks=true,
linkcolor=blue,
filecolor=magenta,      
urlcolor=blue,
citecolor=blue,
}


\title{Lecture 10}
\date{\today}
\author{Michael Brodskiy\\ \small Professor: M. Onabajo}

\begin{document}

\maketitle

\begin{itemize}

  \item Some Important Facts

    \begin{itemize}

      \item Intrinsic (pure) silicon: number of electrons = number of holes

        \begin{itemize}

          \item Equal electron and hold concentrations ($n_i=n_p$)

        \end{itemize}

    \end{itemize}

  \item Band Gap Energy

    \begin{itemize}

      \item Band gap energy is the energy required to break a covalent bond and to free an electron

        \begin{itemize}

          \item $E_g=.66[\si{\eV}]$ germanium

          \item $E_g=1.12[\si{\eV}]$ silicon

          \item $E_g=3.36[\si{\eV}]$ gallium nitride, used in LEDs

        \end{itemize}

      \item Metals have $E_g=0$

        \begin{itemize}

          \item Very large number of free electrons $\to$ high conductance

        \end{itemize}

      \item Insulators have $E_g>5[\si{\eV}]$

        \begin{itemize}

          \item Almost NO free electrons $\to$ zero conductance

        \end{itemize}

    \end{itemize}

  \item Doping

    \begin{itemize}

      \item The intentional addition of impurities to a semiconductor to create more free electrons or more holes

        \begin{itemize}

          \item Creates extrinsic material

        \end{itemize}

      \item $n$-type material

        \begin{itemize}

          \item More electrons than holes ($n>p$)

        \end{itemize}

      \item $p$-type material

        \begin{itemize}

          \item More holes than electrons ($p>n$)

        \end{itemize}

    \end{itemize}

  \item $n$-type Silicon

    \begin{itemize}

      \item Elements in column V of the periodic table have 5 valence electrons

      \item Example: Phosphorus (P) is a popular electron donor

      \item The phosphorus atom has donated an electron to the semiconductor

        \begin{itemize}

          \item Column V atoms are called donors

        \end{itemize}

      \item The phosphorus is missing one of its electrons, so it has a positive charge (+1) $\to$ phosphorus ion (P+)

        \begin{itemize}

          \item P+ is bound to the silicon $\to$ this +1 charge can not move

        \end{itemize}

      \item The extra electron concentration is equal to the concentration of donor atoms ($N_d$)

      \item The overall net charge is zero, requiring that:

        $$n=p+N_d$$

    \end{itemize}

  \item $p$-type Silicon

    \begin{itemize}

      \item Elements in column III of the periodic table have 3 valence electrons

      \item Example: Boron (B) is a popular electron acceptor

        \begin{itemize}

          \item 3 electrons (from the boron) form covalent bonds with the Si

          \item A 4th electron is needed for each boron atom

            \begin{itemize}

              \item This 4th electron creates a hole

            \end{itemize}

        \end{itemize}

      \item The boron atom has accepted an electron from the semiconductor $\to$ column III atoms are called acceptors

      \item The boron atom has one extra electron, so that it has a negative charge (-1) $\to$ boron ion (B-)

    \end{itemize}

  \item Additional info for doped Si

    \begin{itemize}

      \item The mass-action law hold for both $n$ and $p$-type Si

        \begin{itemize}

          \item After doping, the increased electron concentration makes hole recombination more likely (and vice versa)

          \item $pn=p_in_i$

          \item $\sin(p_i)=n_i$ in intrinsic Si: $pn=n_i^2$ (at a specific temperature)

        \end{itemize}

    \end{itemize}

  \item The pn Junction

    \begin{itemize}

      \item At zero bias ($V_D=0$), neither electrons not holes can overcome the built-in voltage barrer of $\approx.7[\si{\volt}]$ (for Si), so $I_D=0$

      \item As the bias ($V_D$) increases toward $.7[\si{\volt}]$, the electrons and holes can overcome the built-in voltage barrier ($I_D>0$)

      \item As the bias ($V_D$) becomes negative, the barrier becomes larger

        \begin{itemize}

          \item Only electrons and holes due to broken bonds in the depletion region contribute to the diode currnt ($I_D=I_s$)

        \end{itemize}

      \item As $V_D$ becomes very negative, the barrier increases even more

        \begin{itemize}

          \item Free electrons and holes (due to broken bonds) in the depletion region are accelerated to high energy ($>E_g$)

            \begin{itemize}

              \item Breaking of other covalent bonds

            \end{itemize}

        \end{itemize}

    \end{itemize}

  \item Junction Capacitance

    \begin{itemize}

      \item Parasitic (unwanted) junction capacitance: $C\approx \varepsilon A/W$), where:

        \begin{itemize}

          \item The depletion layer width ($W$) depends on the bias voltage

          \item $\varepsilon\approx \varepsilon_r\varepsilon_o$ ($=11.9\varepsilon_o$ for Si)

            \begin{itemize}

              \item $\varepsilon_r$ is the relative dielectric constant (permittivity) of a material

              \item Dielectric constant for a vacuum: $\varepsilon_o\approx 8.85\cdot10^{-12}[\si{\farad}/\si{\meter}]$

            \end{itemize}

        \end{itemize}

    \end{itemize}

  \item Depletion Capacitance

    $$C_j=\frac{C_{j0}}{[1-(V_{DQ}/\Phi_o)]^m}$$

    \begin{itemize}

      \item $C_{j0}$ is the incremental depletion capacitance from zero bias

      \item $V_{DQ}$ is the operating point ($Q$-point) voltage (DC bias voltage)

      \item $M$ is the grading coefficient (typically 1/3 to 1/2)

      \item $\Phi_o$ is the built-in barrier potential

      \item $C_j$ models the depletion capacitance with bias dependence

        \begin{itemize}

          \item Note: $C_j$ is a small-signal parameter (most accurate for small AC signal changes around a fixed DC Q-point)

        \end{itemize}

    \end{itemize}

\end{itemize}

\end{document}

