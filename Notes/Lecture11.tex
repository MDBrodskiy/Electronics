%%%%%%%%%%%%%%%%%%%%%%%%%%%%%%%%%%%%%%%%%%%%%%%%%%%%%%%%%%%%%%%%%%%%%%%%%%%%%%%%%%%%%%%%%%%%%%%%%%%%%%%%%%%%%%%%%%%%%%%%%%%%%%%%%%%%%%%%%%%%%%%%%%%%%%%%%%%%%%%%%%%
% Written By Michael Brodskiy
% Class: Fundamentals of Electronics
% Professor: M. Onabajo
%%%%%%%%%%%%%%%%%%%%%%%%%%%%%%%%%%%%%%%%%%%%%%%%%%%%%%%%%%%%%%%%%%%%%%%%%%%%%%%%%%%%%%%%%%%%%%%%%%%%%%%%%%%%%%%%%%%%%%%%%%%%%%%%%%%%%%%%%%%%%%%%%%%%%%%%%%%%%%%%%%%

\include{Includes.tex}

\title{Lecture 11}
\date{\today}
\author{Michael Brodskiy\\ \small Professor: M. Onabajo}

\begin{document}

\maketitle

\begin{itemize}

  \item npn Bipolar Junction Transistors

    \begin{itemize}

      \item The collector ``collects'' electrons, and causes current to flow through the emitter

    \end{itemize}
    
  \item npn Structure without Bias

    \begin{itemize}

      \item At zero bias ($V_{be}=V_{bc}=0$), neither electrons nor holes can overcome this built-in voltage barrier of $\approx.7[\si{\volt}]$ (for Si)

        \begin{itemize}

          \item $I_B=I_C=0$ (cutoff)

        \end{itemize}

    \end{itemize}

  \item npn Structure with Forward-Biased EBJ

    \begin{itemize}

      \item When ($V_{be}=.65[\si{\volt}]$, $V_c>V_b$), electrons and holes can overcome the built-in voltage barrier between the base and emitter

        \begin{itemize}

          \item $I_b>0$ and $I_e>I_b$ (due to $n^+$ emitter doping)

        \end{itemize}

      \item If the base region is very thin, the electrons injected by the emitter are collected by the positive voltage applied at $V_c$

        \begin{itemize}

          \item $I_c\approx I_E >> I_B$ (active region)

        \end{itemize}

      \item If the base region is too thick, many electrons injected at the emitter are lost by recombining with holes in the base before the voltage applied at $V_c$ can collect them

        \begin{itemize}

          \item $I_c<I_E$ (active region with low $\alpha$ and $\beta\to$ low gain)

        \end{itemize}

    \end{itemize}

  \item Achievement of high $\beta$ during Fabrication

    \begin{itemize}

      \item Thin base region

        \begin{itemize}

          \item Increases the collection efficiency for injected electrons

          \item Reduces the chance of electron recombination in the base

        \end{itemize}

      \item Heavily-doped emitter

        \begin{itemize}

          \item $I_E/I_B\propto n (\text{emitter})/ p (\text{base})\propto \beta$

        \end{itemize}

      \item Doping concentrations are difficult to control precisely

        \begin{itemize}

          \item Current gain is not uniform among BJTs (exception: when the BJTs are all fabricated on the same integrated circuit $\to$ small variations)

        \end{itemize}

    \end{itemize}

  \item The Early Effect

    \begin{itemize}

      \item As $V_c$ increases, the depletion width of the B-C junction widens

        \begin{itemize}

          \item Base width becomes more narrow

          \item Increased collection efficiency

          \item Finally, $I_c/I_b$ increases (higher $\beta$)

        \end{itemize}

    \end{itemize}

  \item Summary: Regions of Operation

    \begin{itemize}

      \item CBJ: Collector-Base Junction

      \item EBJ: Emitter-Base Junction

      \item RB: Reverse-Biased

      \item FB: Forward-Biased

        \begin{center}
          \begin{tabular}[H]{|c|c|c|}
            \hline
            \rowcolor{black} \textcolor{white}{EBJ} & \textcolor{white}{CBJ} & \textcolor{white}{Mode}\\
            \hline
            \rowcolor{black!5} FB & RB & Active\\
            \hline
            \rowcolor{black!30} FB & FB & Saturation\\
            \hline
            \rowcolor{black!5} RB & RB & Cutoff\\
            \hline
          \end{tabular}
        \end{center}

      \item When designing amplifers: make sure the BJT is in active mode

    \end{itemize}

  \item npn BJT Model for the Active Region

    \begin{itemize}

      \item Current gain: $\beta=I_C/I_B$ (range: 10-1000, typical: 100-200)

      \item Model usage:

        \begin{itemize}

          \item Assume (gues) a region of operation

          \item Replace the BJT with the appropriate model

          \item Perform standard circuit analysis

          \item Check that the conditions for the region of operation are met (if the conditions are not met: change the accumption and repeat the analysis)

          \item From KVL: $V_{BC}=V_{BE}-V_{CE}$

          \item Assuming $V_{BE}=.7$ (forward-biased)

            \begin{itemize}

              \item $V_{BC}=V_{BE}-V_{CE(min)}=.7-.2=.5[\si{\volt}]<.7[\si{\volt}]$ (ensures that CBJ remains reverse-biased)

              \item $V_{CE\downarrow}\to V_{BC\uparrow} \to$ CBJ becomes forward-biased $\to$ saturation

              \item Note: in some models (other books), $V_{CE}>.3[\si{\volt}]$ is assumed as condition for the active mode of operation

            \end{itemize}

        \end{itemize}

    \end{itemize}

  \item npn BJT Model for the Saturation Region

    \begin{itemize}

      \item Forward Voltage Drop of $.7[\si{\volt}]$ [$V_{BE}$] (note for all models: depends on BJT type, some models use different assumptions, such as $.6[\si{\volt}]$)

    \end{itemize}

  \item npn BJT Model for the Cutoff Region

    \begin{itemize}

      \item BJT forms an open circuit since both diodes are in reverse bias

    \end{itemize}

  \item Key Equations:

    $$I_E=I_C+I_B$$

    \begin{itemize}

      \item When the base-emitter junction is forward-biased:

        $$I_E=I_{ES}[e^{V_{BE}/V_T}-1]$$

      \item Valid only in the active region:

        $$I_C=\alpha I_{E}$$
        $$I_C=\alpha I_{ES}[e^{V_{BE}/V_T}-1]$$
        $$I_B=(1-\alpha)I_E$$
        $$I_C=\beta I_B$$
        $$I_S=\alpha I_{ES}$$
        $$\beta=\frac{\alpha}{1-\alpha}\quad\text{ and }\quad \alpha=\frac{\beta}{1+\beta}$$

      \item Typical (with high $\beta$): $\alpha$ is near unity

    \end{itemize}

\end{itemize}

\end{document}

