%%%%%%%%%%%%%%%%%%%%%%%%%%%%%%%%%%%%%%%%%%%%%%%%%%%%%%%%%%%%%%%%%%%%%%%%%%%%%%%%%%%%%%%%%%%%%%%%%%%%%%%%%%%%%%%%%%%%%%%%%%%%%%%%%%%%%%%%%%%%%%%%%%%%%%%%%%%%%%%%%%%
% Written By Michael Brodskiy
% Class: Fundamentals of Electronics
% Professor: M. Onabajo
%%%%%%%%%%%%%%%%%%%%%%%%%%%%%%%%%%%%%%%%%%%%%%%%%%%%%%%%%%%%%%%%%%%%%%%%%%%%%%%%%%%%%%%%%%%%%%%%%%%%%%%%%%%%%%%%%%%%%%%%%%%%%%%%%%%%%%%%%%%%%%%%%%%%%%%%%%%%%%%%%%%

\documentclass[12pt]{article} 
\usepackage{alphalph}
\usepackage[utf8]{inputenc}
\usepackage[russian,english]{babel}
\usepackage{titling}
\usepackage{amsmath}
\usepackage{graphicx}
\usepackage{enumitem}
\usepackage{amssymb}
\usepackage[super]{nth}
\usepackage{everysel}
\usepackage{ragged2e}
\usepackage{geometry}
\usepackage{multicol}
\usepackage{fancyhdr}
\usepackage{cancel}
\usepackage{siunitx}
\usepackage{physics}
\usepackage{tikz}
\usepackage{mathdots}
\usepackage{yhmath}
\usepackage{cancel}
\usepackage{color}
\usepackage{array}
\usepackage{multirow}
\usepackage{gensymb}
\usepackage{tabularx}
\usepackage{extarrows}
\usepackage{booktabs}
\usepackage{lastpage}
\usetikzlibrary{fadings}
\usetikzlibrary{patterns}
\usetikzlibrary{shadows.blur}
\usetikzlibrary{shapes}

\geometry{top=1.0in,bottom=1.0in,left=1.0in,right=1.0in}
\newcommand{\subtitle}[1]{%
  \posttitle{%
    \par\end{center}
    \begin{center}\large#1\end{center}
    \vskip0.5em}%

}
\usepackage{hyperref}
\hypersetup{
colorlinks=true,
linkcolor=blue,
filecolor=magenta,      
urlcolor=blue,
citecolor=blue,
}


\title{Lecture 5}
\date{\today}
\author{Michael Brodskiy\\ \small Professor: M. Onabajo}

\begin{document}

\maketitle

\begin{itemize}

  \item Finite Open-Loop Gain and Bandwidth

    \begin{itemize}

      \item Assumptions during most previous examples

        \begin{itemize}

          \item Infinite $A_{0OL}$ (subscript ``0'' indicates DC gain)

          \item Gain independence with respect to frequency (``flat gain'')

        \end{itemize}

      \item Open-loop gain of a typical (real) op-amp:

        \begin{itemize}

          \item Single-pole approximation ($F_{BOL}=$ break frequency)

          \item High-frequency roll-off with $-20\text{dB/dec}$ (single-pole approximation — for first order filters, can approximate $f_{t}=A_{0OL}f_{BOL}$)

          \item $f_t=$ transition frequency (unity-gain)

        \end{itemize}

    \end{itemize}

  \item Mathematical Representation of Finite Gain/Bandwidth

    \begin{itemize}

      \item $A_{0OL}=$ DC gain, $\omega_=$ break frequency, $\omega_t=$ unity-gain frequency

      \item Op-amp model with a transfer function of a single-pole low-pass filter:

        $$A(\omega)=\frac{A_{0OL}}{1+(j\omega/\omega_B)},\,|A(\omega)|=\frac{A_{0OL}}{\sqrt{1+\left( \dfrac{\omega}{\omega_B} \right)^2}}$$

    \end{itemize}

  \item Inverting Amplifier Analysis

    \begin{itemize}

      \item High closed-loop gain (high $R_2/R_1$ ratio) reduces the closed-loop break frequency

    \end{itemize}

  \item Closed-Loop Gain versus Break Frequency Trade-off

    \begin{itemize}

      \item Fundamental gain-bandwidth (GBW) product limitation:

        $$f_t=A_{0OL}f_{BOL}=A_{0CL}f_{BCL}$$

      \item When $f>>f_B$:

        $$|A(f)|\approx A_{0OL}\cdot\frac{f_B}{f}=\frac{f_t}{f}$$

    \end{itemize}

  \item Closed-Loop: Gain Bandwidth $\propto f_{3\text{dB}}$

    \begin{itemize}

      \item When an op-amp is connected in a feedback configuration, the gain-bandwidth product ($f_t$) remains unchanged

      \item The 3dB frequency (break frequency) depends on feedback network components

      \item Gain-bandwidth product ($f_t=A_{0OL}f_{BOL}$)

    \end{itemize}

  \item Large-Signal Operation: Voltage Swing

    \begin{itemize}

      \item Output voltage swing limitation

        \begin{itemize}

          \item The output voltage can only be in the following range:

            $$V_{S-}+x<V_o<V_{S+}-x$$

          \item The output limits should be specified in the manufacturers datasheet

        \end{itemize}

      \item Clipping (saturation) occurs if the above condition is not met

    \end{itemize}

  \item Linear Operating Range

    \begin{itemize}

      \item The input/output transfer characteristic of an op-amp (with a specified supply) voltage provide valuable information about large-signal operation

    \end{itemize}

  \item Large-Signal Operation: Current Restrictions

    \begin{itemize}

      \item Op-amps have specified output current limits

      \item The op-amp must source/sink the current to/from load impedance (and feedback network elements)

      \item Careful: Small load or feedback resistors $\to$ high $I_o$


      \item Clipping occurs when $I_o>I_{\text{limit}}$ would be required, but $I_o=I_{\text{limit}}$

    \end{itemize}

  \item Finite Open-Loop Gain and Bandwidth

    $$A(f)=\frac{A_{0OL}}{1+j(f/f_{BOL})}$$

    \begin{itemize}

      \item Closed-Loop Impact of Finite Gain/Bandwidth

        \begin{itemize}

          \item Inverting amplifier: $G=-\frac{R_2}{R_1}$

          \item Non-inverting amplifier: $G=1+\frac{R_2}{R_1}$

          \item For both cases: $f_{3\text{dB}}\approx\frac{f_t}{1+(R_2/R_1)}$

        \end{itemize}

    \end{itemize}

  \item Output Slew-Rate Limitation

    $$\Big|\frac{dv_o(t)}{dt}\Big|\leq SR$$

    \begin{itemize}

      \item The magnitude of the output voltage's rate of change can not exceed the slew-rate ($SR$) specification of the op-amp

      \item Typical $10^5[\si{\volt}/\si{\second}]<SR<10^8[\si{\volt}/\si{\second}]$

      \item Usually the $SR$ is specified with load resistance conditions

    \end{itemize}

  \item DC Imperfections of Op-Amps

    \begin{itemize}

      \item Bias Current ($I_B$)

        \begin{itemize}

          \item Required for proper operation of internal circuitry (or resulting from unwanted leakage currents)

          \item Typical range: $0.1[\si{\nano\ampere}]-1[\si{\micro\ampere}]$

          \item $I_B=(I_{B+}-I_{B-})/2$

          \item $I_{B+}$ and $I_{B-}$ flow at the respective terminals

        \end{itemize}

      \item Offset Current ($I_{off}$)

        \begin{itemize}

          \item $|\pm I_{off}|=|I_{B+}-I_{B-}|<200[\si{\nano\ampere}]$ normally

          \item Results from internal device mismatches (transistors, resistors, etc.)

        \end{itemize}

      \item Offset Voltage ($V_{off}$)

        \begin{itemize}

          \item Due to internal device mismatches

          \item $|\pm V_{off}|<$ a few millivolts

        \end{itemize}

    \end{itemize}

  \item Analysis Procedure with DC Imperfections

    \begin{itemize}

      \item Draw a schematic diagram in which the source for a single DC imperfection is included (modeled)

      \item Replace all other sources

        \begin{itemize}

          \item Voltage source $\to$ short circuit

          \item Current source $\to$ open circuit

        \end{itemize}

      \item Follow standard circuit analysis laws and ideal op-amp assumptions

    \end{itemize}

\end{itemize}

\end{document}

