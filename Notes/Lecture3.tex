%%%%%%%%%%%%%%%%%%%%%%%%%%%%%%%%%%%%%%%%%%%%%%%%%%%%%%%%%%%%%%%%%%%%%%%%%%%%%%%%%%%%%%%%%%%%%%%%%%%%%%%%%%%%%%%%%%%%%%%%%%%%%%%%%%%%%%%%%%%%%%%%%%%%%%%%%%%%%%%%%%%
% Written By Michael Brodskiy
% Class: Fundamentals of Electronics
% Professor: M. Onabajo
%%%%%%%%%%%%%%%%%%%%%%%%%%%%%%%%%%%%%%%%%%%%%%%%%%%%%%%%%%%%%%%%%%%%%%%%%%%%%%%%%%%%%%%%%%%%%%%%%%%%%%%%%%%%%%%%%%%%%%%%%%%%%%%%%%%%%%%%%%%%%%%%%%%%%%%%%%%%%%%%%%%

\documentclass[12pt]{article} 
\usepackage{alphalph}
\usepackage[utf8]{inputenc}
\usepackage[russian,english]{babel}
\usepackage{titling}
\usepackage{amsmath}
\usepackage{graphicx}
\usepackage{enumitem}
\usepackage{amssymb}
\usepackage[super]{nth}
\usepackage{everysel}
\usepackage{ragged2e}
\usepackage{geometry}
\usepackage{multicol}
\usepackage{fancyhdr}
\usepackage{cancel}
\usepackage{siunitx}
\usepackage{physics}
\usepackage{tikz}
\usepackage{mathdots}
\usepackage{yhmath}
\usepackage{cancel}
\usepackage{color}
\usepackage{array}
\usepackage{multirow}
\usepackage{gensymb}
\usepackage{tabularx}
\usepackage{extarrows}
\usepackage{booktabs}
\usepackage{lastpage}
\usetikzlibrary{fadings}
\usetikzlibrary{patterns}
\usetikzlibrary{shadows.blur}
\usetikzlibrary{shapes}

\geometry{top=1.0in,bottom=1.0in,left=1.0in,right=1.0in}
\newcommand{\subtitle}[1]{%
  \posttitle{%
    \par\end{center}
    \begin{center}\large#1\end{center}
    \vskip0.5em}%

}
\usepackage{hyperref}
\hypersetup{
colorlinks=true,
linkcolor=blue,
filecolor=magenta,      
urlcolor=blue,
citecolor=blue,
}


\title{Lecture 3}
\date{\today}
\author{Michael Brodskiy\\ \small Professor: M. Onabajo}

\begin{document}

\maketitle

\begin{itemize}

  \item Frequency Dependence (Impedance)

    \begin{itemize}

      \item Capacitor

        $$Z_c=\frac{1}{j\omega C}$$

      \item Inductors

        $$Z_L=j\omega L=SL$$

      \item Note, for capacitors impedance decreases with frequency, while it increases with frequency for inductors

    \end{itemize}

  \item DC Coupling
    
    \begin{itemize}

      \item Amplifier stages are directly connected together

      \item High-frequency gain decreases (``rolls off'') due to unwanted (``parasitic'') capacitances and inductances

    \end{itemize}

  \item AC Coupling

    \begin{itemize}

      \item Input-coupling capacitors are sometimes referred to as DC-blocking Capacitors

      \item Improved isolation between stages because the capacitors ``block'' DC current/voltages ($Z_c=1/j\omega C\to$ infinite impedance at $\omega=0$)

      \item Impacts the low-frequency response

    \end{itemize}

  \item Impact of Parasitics (Stray Inductances/Capacitances)

    \begin{itemize}

      \item Stray inductances/capacitances (often called ``parasitics'') result from non-ideal properties of materials:

        \begin{itemize}

          \item Integrated circuits, chip packages, printed circuit boards, cables, \ldots

        \end{itemize}

      \item High-frequency gain reduction from:

        \begin{itemize}

          \item Capacitors in parallel with the signal path

          \item Inductors in series with the signal path

        \end{itemize}

      \item Computer-based simulations are used for complex models and circuits

    \end{itemize}

  \item Half-Power Bandwidth

    \begin{itemize}

      \item $P_o=(AV_{\text{inRMS}})^2/R_L\to P_o=P_{max}/2\text{ when} A=A_{max}/\sqrt{2}$

      \item By convention, the frequencies $f_H$ and $f_L$ at which $P_o=P_{max}/2$ are referred to as half-power frequencies or $-3\text{db}$ frequencies

        \begin{itemize}

          \item Note: $20\log(A_{max}/\sqrt{2})=20\log(A_{max})-20\log(\sqrt{2})=A_{max\text{(dB)}}-3.01\text{dB}$

        \end{itemize}

      \item Amplifier bandwidth: $B=f_H-f_L$

    \end{itemize}

  \item Complex Gain, Frequency Response

    \begin{itemize}

      \item Complex transfer function $T(j\omega)$

        \begin{itemize}

          \item $s=j\omega=j(2\pi f)\to T(s)=\frac{V_o(s)/V_i(s)}$

          \item Frequency-dependent gain and phase

          \item $|T|\angle\phi=R+jX$, where $|T|=(R^2+X^2)^{1/2}$, $\phi=\tan^{-1}\left( \frac{X}{R} \right)$

        \end{itemize}

    \end{itemize}

  \item First-Order Low-Pass Filter

    \begin{itemize}

      \item $V_o(s)=V_i(s)\frac{Z_c}{Z_c+Z_r}$, where $Z_r=R$, $Z_c=\frac{1}{sC}=\frac{1}{j\omega C}$

      \item $T(j\omega)=\dfrac{V_o(j\omega)}{V_i(j\omega)}=\dfrac{1/(j\omega C)}{(1/j\omega C)+R}=\dfrac{1}{1+j\omega RC}$

      \item Let $\omega_o=(1/RC)=(1/\tau)$ and $K=1$

        \begin{itemize}

          \item $\tau=RC$ is the only time constant of this circuit with a single pole formed by the resistor and capacitor

          \item $T(j\omega)=\dfrac{K}{[1+j\frac{\omega}{\omega_o}]}$

        \end{itemize}

    \end{itemize}

  \item Transfer Function Normalization (First-Order LPF Case)

    \begin{itemize}

      \item Typically, $K\neq1$

      \item When normalizing a magnitude response, plot: $|T(j\omega)/K|$

        \begin{itemize}

          \item $20\log(|T(j\omega)/K|)=20\log(1)=0[\text{dB}]$ becomes max gain

        \end{itemize}

      \item Low-pass filter characteristics:

        \begin{itemize}

          \item For $\omega <<\omega_o$: $|T(j\omega)/K|\approx 1$ (0[dB])

          \item For $\omega >>\omega_o$: $|T(j\omega)/K|\approx \dfrac{\omega_o}{\omega}\to$ high-frequency roll-off

          \item Slope is -20[dB]/decade (or -6[dB]/octave)

        \end{itemize}

    \end{itemize}

  \item Bode Plot of the Low-Pass Filter

    \begin{itemize}

      \item Attenuates high-frequency signal components

      \item ``Corner frequency'' $\leftrightarrow$ -3[dB] frequency is the ``cutoff frequency''

        \begin{itemize}

          \item Often labeled $f_c(\omega_c)$, $f_{3\text{dB}}$, $(\omega_{3\text{dB}})$, $f_o(\omega_o)$, or $f_B(\omega_B)$

          \item In the LPF case, the corner frequency is often called ``bandwidth of the filter''

        \end{itemize}

    \end{itemize}

  \item First-Order High Pass Filter

    $$T(j\omega)=\frac{1}{1-j(\omega_o/\omega)}$$

    \begin{itemize}

      \item Where $\omega_o=1/(RC)=(1/\tau)$, with $\tau=RC$ as the time constant

      \item In general:

        $$T(j\omega)=\frac{K}{1-j(\omega_o/\omega)}$$

        \begin{itemize}

          \item As $\omega\to 0$, $T(j\omega\to)$ (low frequency rejection)

          \item As $\omega\to \infty$, $T(j\omega\to K)$ (high frequency transmission)

        \end{itemize}

    \end{itemize}

  \item Bode Plot of the High-Pass Filter

    \begin{itemize}

      \item Attenuates low-frequency components

      \item ``Corner frequency'' $\leftrightarrow$ -3[dB] frequency is the ``cutoff frequency''

        \begin{itemize}

          \item Often labeled $f_c(\omega_c)$, $f_{3\text{dB}}$, $(\omega_{3\text{dB}})$

          \item $f_{3\text{dB}}\neq$ bandwidth

        \end{itemize}

    \end{itemize}

  \item Bandpass (Mid-Band) Filter

    \begin{itemize}

      \item Attenuates signal components outside of bandwidth

      \item Bandwidth: $B=f_{c(LP)}-f_{c(HP)}=f_{\text{High3DB}}-f_{\text{Low3dB}}$

    \end{itemize}

  \item Ideal Operational Amplifiers (Op-Amps)

    \begin{itemize}

      \item Infinite open-loop differential gain $A_{dOL}=V_o/(V_+-V_-)$

      \item Infinite input impedance ($R_i=\infty$, $i_{in+}=i_{in-}=0$)

      \item Zero output impedance ($R_o=0$)

      \item Zero common-mode gain (CMRR=$A_{dOL}/A_{cm}=\infty$)

      \item Infinite bandwidth (no high or low frequency gain roll-off)

    \end{itemize}

  \item The Summing-Point Constraint

    \begin{itemize}

      \item Only applies when the op-amp is used in negative feedback, which is often the case

      \item Assuming the ideal op-amp (in particular: $A_{dOL}=\infty$), the feedback action forces $V_+-V_-=0$ (a virtual short-circuit between the terminals)

        \begin{itemize}

          \item No current flow into the input terminals

        \end{itemize}

    \end{itemize}

\end{itemize}

\end{document}

