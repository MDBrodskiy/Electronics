%%%%%%%%%%%%%%%%%%%%%%%%%%%%%%%%%%%%%%%%%%%%%%%%%%%%%%%%%%%%%%%%%%%%%%%%%%%%%%%%%%%%%%%%%%%%%%%%%%%%%%%%%%%%%%%%%%%%%%%%%%%%%%%%%%%%%%%%%%%%%%%%%%%%%%%%%%%%%%%%%%%
% Written By Michael Brodskiy
% Class: Fundamentals of Electronics
% Professor: M. Onabajo
%%%%%%%%%%%%%%%%%%%%%%%%%%%%%%%%%%%%%%%%%%%%%%%%%%%%%%%%%%%%%%%%%%%%%%%%%%%%%%%%%%%%%%%%%%%%%%%%%%%%%%%%%%%%%%%%%%%%%%%%%%%%%%%%%%%%%%%%%%%%%%%%%%%%%%%%%%%%%%%%%%%

\documentclass[12pt]{article} 
\usepackage{alphalph}
\usepackage[utf8]{inputenc}
\usepackage[russian,english]{babel}
\usepackage{titling}
\usepackage{amsmath}
\usepackage{graphicx}
\usepackage{enumitem}
\usepackage{amssymb}
\usepackage[super]{nth}
\usepackage{everysel}
\usepackage{ragged2e}
\usepackage{geometry}
\usepackage{multicol}
\usepackage{fancyhdr}
\usepackage{cancel}
\usepackage{siunitx}
\usepackage{physics}
\usepackage{tikz}
\usepackage{mathdots}
\usepackage{yhmath}
\usepackage{cancel}
\usepackage{color}
\usepackage{array}
\usepackage{multirow}
\usepackage{gensymb}
\usepackage{tabularx}
\usepackage{extarrows}
\usepackage{booktabs}
\usepackage{lastpage}
\usetikzlibrary{fadings}
\usetikzlibrary{patterns}
\usetikzlibrary{shadows.blur}
\usetikzlibrary{shapes}

\geometry{top=1.0in,bottom=1.0in,left=1.0in,right=1.0in}
\newcommand{\subtitle}[1]{%
  \posttitle{%
    \par\end{center}
    \begin{center}\large#1\end{center}
    \vskip0.5em}%

}
\usepackage{hyperref}
\hypersetup{
colorlinks=true,
linkcolor=blue,
filecolor=magenta,      
urlcolor=blue,
citecolor=blue,
}


\title{Lecture 4}
\date{\today}
\author{Michael Brodskiy\\ \small Professor: M. Onabajo}

\begin{document}

\maketitle

\begin{itemize}

  \item Ideal Op-Amp Summing Constraint

    \begin{itemize}

      \item Only applies when the op-amp is used in negative feedback, which is often the case

    \end{itemize}

  \item Ideal Op-Amp Circuit Analysis Procedure

    \begin{itemize}

      \item Check that the op-amp is connected with negative feedback

      \item Assume $V_+-V_-=0$ based on the summing node constraint

      \item Apply standard circuit analysis techniques (KVL, KCL, Ohm's Law)

      \item For an inverting amplifier:

        $$i_i=V_{i}/R_1\text{, and from KCL we obtain }i_2=i_1$$
        $$\text{From KVL: }V_o=-i_2R_2$$

      \item For non-inverting amplifiers:

        $$A_v=\frac{V_o}{V_i}=1+\frac{R_2}{R_1}$$
        $$R_i=\frac{V_i}{i_i}=\infty\text{ (with ideal op-amp model)}$$
        $$R_o=0\text{ (with ideal op-amp model)}$$

    \end{itemize}

  \item Differential Amplifier

    \begin{itemize}

      \item A few observations

        $$i_2=-i_1\text{ (per summing-point constraint)}$$

      \item The voltage division principles give the following equation:

        $$V_+=V_2\left( \frac{R_2}{R_1+R_2} \right)\text{ and }V_+=V_-$$

      \item Employing KCL, we can calculate:

        $$\frac{R_2}{R_1}V_1-V_o=\frac{R_1+R_2}{R_1}V_+$$
        $$V_o=\frac{R_2}{R_1}(V_2-V_1)$$

      \item The differential gain becomes:

        $$A_{vd}=\frac{V_o}{(V_2-V_1)}=\frac{R_2}{R_1}$$

      \item The common-mode gain is evaluated with $V_1=V_2=V_{icm}$:

        $$V_{ocm}=(R_2/R_1)(V_2-V_1)\to A_{cm}=\frac{V_{ocm}}{V_{icm}}=0$$

      \item The CMRR becomes $\infty$

    \end{itemize}

  \item Voltage Follower

    \begin{itemize}

      \item $V_o=V_i$ per summing point constraint

        $$A_v=(V_o/V_i)=1$$

      \item Also called ``unity gain buffer''

        $$R_i=\infty$$
        $$R_o=0$$

      \item A good circuit to couple amplifier stages together with reduced loading effects:

        \begin{itemize}

          \item High $R_i$ regardless of $R_1$

        \end{itemize}

    \end{itemize}

  \item Finite Open-Loop Gain

    \begin{itemize}

      \item In practice, the open-loop gain ($A_{OL}$) is 60-120dB

      \item Degraded summing point quality: $V_x=V_+-V_-\neq0\to V_z=\frac{V_o}{A_{OL}}$

      \item Feedback factor for this circuit: $\beta=R_1/(R_1+R_2)$ occurs due to voltage division

        \begin{itemize}

          \item $\beta$ is the fraction of the output that is fed back to the $V_-$ terminal

        \end{itemize}

        $$V_o=(V_+-V_-)A_{OL}=-V_-A_{OL}\to V_-=\frac{-V_o}{A_{OL}}$$

    \end{itemize}

\end{itemize}

\end{document}

