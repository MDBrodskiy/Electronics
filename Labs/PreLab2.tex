%%%%%%%%%%%%%%%%%%%%%%%%%%%%%%%%%%%%%%%%%%%%%%%%%%%%%%%%%%%%%%%%%%%%%%%%%%%%%%%%%%%%%%%%%%%%%%%%%%%%%%%%%%%%%%%%%%%%%%%%%%%%%%%%%%%%%%%%%%%%%%%%%%%%%%%%%%%%%%%%%%%
% Written By Michael Brodskiy
% Class: Fundamentals of Electronics
% Professor: I. Salama
%%%%%%%%%%%%%%%%%%%%%%%%%%%%%%%%%%%%%%%%%%%%%%%%%%%%%%%%%%%%%%%%%%%%%%%%%%%%%%%%%%%%%%%%%%%%%%%%%%%%%%%%%%%%%%%%%%%%%%%%%%%%%%%%%%%%%%%%%%%%%%%%%%%%%%%%%%%%%%%%%%%

\documentclass[12pt]{article} 
\usepackage{alphalph}
\usepackage[utf8]{inputenc}
\usepackage[russian,english]{babel}
\usepackage{titling}
\usepackage{amsmath}
\usepackage{graphicx}
\usepackage{enumitem}
\usepackage{amssymb}
\usepackage[super]{nth}
\usepackage{everysel}
\usepackage{ragged2e}
\usepackage{geometry}
\usepackage{multicol}
\usepackage{fancyhdr}
\usepackage{cancel}
\usepackage{siunitx}
\usepackage{physics}
\usepackage{tikz}
\usepackage{mathdots}
\usepackage{yhmath}
\usepackage{cancel}
\usepackage{color}
\usepackage{array}
\usepackage{multirow}
\usepackage{gensymb}
\usepackage{tabularx}
\usepackage{extarrows}
\usepackage{booktabs}
\usepackage{lastpage}
\usetikzlibrary{fadings}
\usetikzlibrary{patterns}
\usetikzlibrary{shadows.blur}
\usetikzlibrary{shapes}

\geometry{top=1.0in,bottom=1.0in,left=1.0in,right=1.0in}
\newcommand{\subtitle}[1]{%
  \posttitle{%
    \par\end{center}
    \begin{center}\large#1\end{center}
    \vskip0.5em}%

}
\usepackage{hyperref}
\hypersetup{
colorlinks=true,
linkcolor=blue,
filecolor=magenta,      
urlcolor=blue,
citecolor=blue,
}


\title{Pre-Lab 2}
\date{\today}
\author{Michael Brodskiy\\ \small Professor: M. Onabajo}

\begin{document}

\maketitle

\begin{enumerate}

  \item

    Using the equation:

    $$i=I_s\left( e^{v/(nV_T)}-1 \right)$$

    and the given measurements, we may write:

    $$10^{-3}=I_s\left( e^{.62/(n.025)}-1 \right)$$
    $$10^{-2}=I_s\left( e^{.69/(n.025)}-1 \right)$$

    Since the exponential is greater than 1, we may simplify to:

    $$\frac{10^{-3}}{e^{24.8/n}}=I_s$$
    $$\frac{10^{-2}}{e^{27.6/n}}=I_s$$

    $$I_s=.001e^{-\frac{24.8}{n}}$$
    $$I_s=.01e^{-\frac{27.6}{n}}$$

    We can combine the two equations to solve for $n$ and $I_s$:

    $$.001e^{-\frac{24.8}{n}}=.01e^{-\frac{27.6}{n}}$$
    $$e^{-\frac{24.8}{n}}=10e^{-\frac{27.6}{n}}$$
    $$e^{\frac{27.6-24.8}{n}}=10$$
    $$\frac{27.6-24.8}{n}=\ln(10)$$
    $$n=\frac{2.8}{\ln(10)}$$
    $$\boxed{n=1.216}$$

    We then plug this into the earlier equations to find $I_s$:

    $$I_s=.001e^{-24.8/1.216}$$
    $$\boxed{I_s=1.3889\cdot10^{-12}[\si{\ampere}]}$$

    We can verify by using the second equation for $I_s$:

    $$I_s=.01e^{-27.6/1.216}$$
    $$\boxed{I_s=1.3889\cdot10^{-12}[\si{\ampere}]}$$

    It is difficult to measure the current in a reverse bias mode, however, because the diode may reach breakdown voltage by increasing the reverse voltage.

  \item Read through, no questions \textcolor{green}{\checkmark}

  \item

\end{enumerate}

\end{document}

